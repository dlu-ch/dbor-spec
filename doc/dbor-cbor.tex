% SPDX-License-Identifier: CC-BY-4.0
% DBOR specification - Dense Binary Object Representation
% Copyright (C) 2020 Daniel Lutz <dlu-ch@users.noreply.github.com>

\section{Comparison to CBOR}
%%%%%%%%%%%%%%%%%%%%%%%%%%%%
\label{sec:comparisontocbor}

In contrast to CBOR, DBOR prioritizes simplicity, precision and efficiency over extensibility.%
\footnote{%
    The limitations of existing implementations of CBOR encoders and decoders suggest
    that the complexity and ambiguity of CBOR is too high.
}

DBOR is focused on the representation of objects and does not try to provide a one-size-fits-all solution
to express context-specific meaning (like "this string represents a date").

\medskip
\begin{BeginParPenalty}
    This leads to the following main differences:
    \begin{itemize}
        \item
        Where feasible, there is only \emph{one} representation for an object,
        and for each representable object there is a unique \emph{canonical} representation.
        For example, there is only \emph{one} well-formed \DborIntegerValue{} that represents the integer~$7$
        (CBOR: 5~different representations with major type 0).

        \item
        Numbers -- integers with an optional base-2 or base-10 factor -- are first-class citizens
        (CBOR: numbers with a base-10 factor require the extension mechanism).

        The number-like object $-0$, $\infty$, $-\infty$ can be used to complement \emph{any} number
        (CBOR: only float).

        \item
        The size of containers is described by the total size of the contained encoded objects
        (CBOR: number of contained items).
        This allows fast forward-traversal.

        Keys in a dictionary must be unique \emph{and ordered}, which makes validation and look-up fast
        (CBOR: keys have to be unique but not ordered, so a decoder has to sort first).

        \item
        The size of encoded objects can be fixed (\DborAllocatorValue).
        This enables the definition of memory layouts and in-place updates of objects without
        following objects being affected.
    \end{itemize}
\end{BeginParPenalty}

\begin{BeginParPenalty}
    Features provided by CBOR but not by DBOR:
    \begin{itemize}
        \item
        \texttt{null}, \texttt{undefined}, \texttt{false}, \texttt{true}:
        these ECMAScript values can be mapped 1:1 to CBOR values
        (DBOR approximations: \DborNoneValue(), \DborNoneValue(), \DborIntegerValue(0), and \DborIntegerValue(1),
        respectively).

        \item Tags:
        the mechanism to modify the meaning of the following object
        (e.g.\ number represented by array).

        \item
        Arrays, maps, byte strings, and text strings of indefinite lengths (for stream encoding).
    \end{itemize}
\end{BeginParPenalty}

DBOR-encoded data is typically significantly smaller than the corresponding CBOR-encoded data.
For each subnormal, non-zero IEEE-754:2008 \texttt{binary16} or \texttt{binary32}, the DBOR-encoded object consists of
\emph{exactly one byte more} than the CBOR-encoded object.
For each other object that both DBOR and CBOR can encode, the DBOR-encoded object consists of \emph{at most as many}
bytes as the CBOR-encoded object.

\begin{BeginParPenalty}
    Examples (smaller one marked with ${}^\diamond$):
    \begin{quote}
        \noindent
        \newcolumntype{R}{>{$}r<{$}}% package 'array'
        \newcommand{\D}{\hbox to 0pt{$\,{}^\diamond$\hss}}
        \begin{tabular}{l Rl Rl}
            \toprule
            & \multicolumn{2}{l}{DBOR} & \multicolumn{2}{l}{CBOR} \\
            \cmidrule(lr){2-3}
            \cmidrule(lr){4-5}
            Object & \text{size} & class & \text{size} & type \\
            \midrule
            $2^{32}$ &
                5\D & \DborIntegerValue & 9 & unsigned int \\
            $2^{64}$ &
                9\D & \DborIntegerValue & 11 & bignum \\[1ex]
            $2^{-3}$ &
                2\D & \DborBinaryRationalValue & 3 & float \\
            $2^{-150}$ &
                6\D & \DborBinaryRationalValue & 9 & float \\
            $2^{-24}$ &
                4 & \DborBinaryRationalValue & 3\D & float \\[1ex]
            $0.1$ &
                2\D & \DborDecimalRationalValue & 4 & decimal fraction \\[1ex]
            - &
                1\D & \DborNoneValue & 3 & null (major type 7) \\
            $\infty$ &
                1\D & \DborInfinityValue & 3 & float \\[1ex]
            \ByteSequence{b_1, ..., b_{256}} &
                258\D & \DborByteStringValue & 259 & byte string \\
            \bottomrule
        \end{tabular}
    \end{quote}
\end{BeginParPenalty}

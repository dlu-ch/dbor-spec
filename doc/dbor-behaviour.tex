% SPDX-License-Identifier: CC-BY-4.0
% DBOR specification - Dense Binary Object Representation
% Copyright (C) 2020 Daniel Lutz <dlu-ch@users.noreply.github.com>

\section{Behaviour}
%%%%%%%%%%%%%%%%%%%
\label{sec:behaviour}

Representing objects as byte sequences (encoding) is only meaningful if there is also a decoding part involved:
a \emph{sender} sends the objects to a \emph{receiver}.

For senders (DBOR encoders) and receivers (DBOR decoders) to be iteroperable, the sender must be able to anticipate the
behaviour of the receiver to some extend.
That's why DBOR specifies certain behavioural aspects of encoders and decoders and how the are exposed by the API
of an implementation.


\subsection{Reserved bytes}
%%%%%%%%%%%%%%%%%%%%%%%%%%%
\label{sec:reservedbytes}

There are 12~numbers that are cannot form the first byte of an (ill-formed or well-formed) DBOR value:
$\IntegerInterval{\HexNumber{F0}}{\HexNumber{FB}}$. These are called \emph{reserved bytes}.

\medskip
A DBOR encoder must never produce a token or value that starts with a reserved byte.

A DBOR decoder must treat each reserved bytes as an invalid or unsupported value of size 1~byte.

\medskip
Applications may prepend or append reserved bytes to DBOR-encoded objects:
\begin{BeginParPenalty}
    Examples:
    \begin{itemize}
        \item
        An application may use $\HexNumber{F0}$ as fill-byte in a buffer that starts with
        DBOR-encoded objects.

        \item
        A binary file that contains DBOR-encoded objects as a byte sequence may have the byte sequence
        \ByteSequence{\HexNumber{F0}, \HexNumber{F0}} prepended to mark it as a non-text file.%
        \footnote{%
           No well-formed UTF-8 string starts with $\HexNumber{F0}$.
           \ByteSequence{\HexNumber{F0}, \HexNumber{F0}} is a valid UTF-16LE and UTF-16BE encoded code point U+F0F0,
           which is part of the Private Use Area.
        }
    \end{itemize}
\end{BeginParPenalty}


\subsection{Statically typed decoder}
%%%%%%%%%%%%%%%%%%%%%%%%%%%%%%%%%%%%%

Decoding%
\footnote{
    Decoding a value does not have to take place in one single operation.

    When decoding a \DborUtfEightStringValue, \DborSequenceValue, or \DborDictionaryValue{} it is in fact recommended to
    check only the size-correctness of the value and return a validating iterator
    (over Unicode code points or container elements).
}
a \DborValue{} in a statically typed environment (e.g. C++ without dynamic polymorphy) is a
type-cast operation from the represented object to a given \emph{target type}.
Such an operation can succeed (the represented value can represented exactly in the target type) or
can fail (the represented value cannot be represented at all in the target type or not exactly).

\medskip
For each of the classes \DborIntegerValue, \DborBinaryRationalValue, \DborDecimalRationalValue, \DborByteStringValue,
and \DborUtfEightStringValue, the decoder must assign an associated set of native data types to represent
(a subset of) its objects.
\begin{BeginParPenalty}
    These sets must be pairwise disjunct:
    \begin{quote}
        \noindent
        \begin{tabular}{l v{0.25\textwidth} v{0.25\textwidth}}
            \toprule
            & \multicolumn{2}{l}{Example} \\
            \cmidrule{2-3}
            Value class & Set of native data types & concrete types for C \\
            \midrule
            \DborIntegerValue & signed and unsigned integers & \texttt{signed int}, \texttt{unsigned int} \\
            \DborBinaryRationalValue & IEEE-754 binary floating point numbers & \texttt{float}, \texttt{double} \\
            \DborDecimalRationalValue & decimal floating point numbers & tuple of two \texttt{int} \\
            \DborByteStringValue & byte sequence & \texttt{const void *} \\
            \DborUtfEightStringValue & character sequence & \texttt{const char *} \\
            \bottomrule
        \end{tabular}
    \end{quote}
\end{BeginParPenalty}

\medskip
With the only exception of numerical conversions, there must be no implicit conversion between
these sets of native types (e.g. between an integer and a character string)

\medskip
\begin{BeginParPenalty}
    A decoder must at least differentiate between these possible decoding results:
    \begin{quote}
        \newcommand{\addextrarowsep}{\addlinespace[1ex]}%
        \noindent
        \begin{tabular}{l p{0.5\textwidth}}
            PRECISE* &
                object is represented exactly in the target type \\ \addextrarowsep
            %
            APPROX-IMPRECISE* &
                object is approximated by a different object representable by the target type \\
            APPROX-EXTREME* &
                object is approximated by the minimum or maximum value representable by the target type
                because it is larger than the maximum or smaller than the minimum
                \\ \addextrarowsep
            %
            RANGE &
                object cannot by represented by the target type, which is in associated set of native data types,
                not even approximately \\
            NO-OBJECT &
                value is \DborNoneValue \\
            UNSUPPORTED &
                decoder does not support the value \\
            %
            INCOMPATIBLE-TYPE &
                target type is not in associated set of native data types \\
            ILL-FORMED &
                value is ill-formed \\
        \end{tabular}
    \end{quote}
\end{BeginParPenalty}

The results marked with * must be accompanied by an instance of the target type.


\subsubsection{Decoding supported \DborNumberValue{} and \DborNumberlikeValue}
%%%%%%%%%%%%%%%%%%%%%%%%%%%%%%%%%%%%%%%%%%%%%%%%%%%%%%%%%%%%%%%%%%%%%%%%%%%%%%

\DborNumberValue{} is special because the sets of objects representable by its subclasses are not pairwise disjunct:
small non-zero integers have a meaning for all of them.
On the other hand, only \DborIntegerValue{} can represent~$0$.

\medskip
Decoding a \DborNumberValue{} or \DborNumberlikeValue{} into a (numerical) native data type is called
a \emph{numerical conversion}.
A \DborNumberValue{} or \DborNumberlikeValue{}~$v$ of type~$V$ is said to be \emph{numerically convertible} into~$T$
if and only if%
\footnote{
    Note that~\DborDecimalRationalValue{} is \emph{not} numerically convertible into any member of an
    associated set of native data types of any other subclass of \DborNumberValue{}.

    This is mainly for performance reasons related to conversions between non-canonical representations
    and to limit the complexity.
}
\begin{itemize}
    \item
    $V$ is a subclass of \DborNumberValue{} and $T$ is a member of the associated set of native data types of~$V$, or

    \item
    $V$ is \DborIntegerValue{} and
    $T$ is a member of the associated set of native data types of \DborBinaryRationalValue{}
    or of \DborDecimalRationalValue{}, or

    \item
    $V$ is a subclass of \DborNumberlikeValue{} and
    $T$ is a member of the associated set of native data types of any subclass of \DborNumberValue{}.

    \item
    $v$ is \DborIntegerValue{}($0$) and
    $T$ is a member of the associated set of native data types of any subclass of \DborNumberValue{}.
\end{itemize}

Let $V$ be a subclass of \DborNumberValue{} or \DborNumberlikeValue{} and
$T$ a native data type that can represent any number
$\in T_{\text{repr}} \subset \SetOfReals \cup \{-\infty, -0, \infty\}$
with $0 \in T$ (e.g. a signed 32~bit integer in two's complement representation).

\begin{enumerate}[a)]
    \item
    If $V$ is numerically convertible into~$T$,
    decoding a well-formed $V$ that is supported by the decoder and
    represents~$v \in T_{\text{repr}}$ into a~$T$ must result in PRECISE with
    the exact representation of $v$ as the returned object.

    \item
    \begin{BeginParPenalty}
        If $V$ is numerically convertible into~$T$,
        decoding a well-formed $V$ that is supported by the decoder and
        represents~$v \notin T_{\text{repr}}$ into a~$T$ must lead to the following results:
        \begin{quote}
            \newcommand{\addextrarowsep}{\addlinespace[0.7ex]}%
            \noindent
            \begin{tabular}{v{0.25\textwidth} l v{0.25\textwidth}}
                \toprule
                Constraints & Result & Returned object as $T$ \\
                \midrule
                %
                $\min{T_{\text{repr}}} \le v \le \max{T_{\text{repr}}}$ &
                    APPROX-IMPRECISE &
                    of all $t \in T_{\text{repr}}$ with minimal $|t - v|$ the one with minimal~$|t|$ \\ \addextrarowsep
                %
                $v < \min{T_{\text{repr}}}$ &
                    APPROX-EXTREME & $\min{T_{\text{repr}}}$ \\ \addextrarowsep
                %
                $v > \max{T_{\text{repr}}}$ &
                    APPROX-EXTREME & $\max{T_{\text{repr}}}$ \\ \addextrarowsep
                \bottomrule
            \end{tabular}
        \end{quote}
    \end{BeginParPenalty}

    \item
    If $V$ is \emph{not} numerically convertible into $T$, decoding a well-formed~$V$ that is supported by the decoder
    must result in INCOMPATIBLE-TYPE.

    \item
    Decoding an ill-formed $V$ that is supported by the decoder must result in ILL-FORMED.
\end{enumerate}


\subsubsection{Decoding other \DborValue}
%%%%%%%%%%%%%%%%%%%%%%%%%%%%%%%%%%%%%%%%%

Decoding a \DborNoneValue{} into any type must result in NO-OBJECT.

\medskip
Decoding a (well-formed or ill-formed) \DborValue{} \emph{not} supported by the decoder
must result in UNSUPPORTED or ILL-FORMED.%
\footnote{%
    The decoder may be unable to check if an unsupported \DborValue{} is well-formed.
}

\medskip
Decoding a well-formed \DborValue{} supported by the decoder that is not a subclass of \DborNumberValue{}
into any type not in its associated set of native data types must result in INCOMPATIBLE-TYPE.

\medskip
Decoding an \DborContainerValue{} or \DborUtfEightStringValue{} that is supported by the decoder and
is not size-correct must result in ILL-FORMED.
If it is size-correct but ill-formed it must result in ILL-FORMED or RANGE
(if another decoding restriction is violated).%
\footnote{%
    Although an \DborUtfEightStringValue{} requires proper UTF-8 encoding for being well-formed,
    a decoder is not required to support multi-byte UTF-8 sequences (i.e. code points $> \HexNumber{7F}$).

    A decoder that \emph{does} supports multi-byte UTF-8 sequences
    may return RANGE when decoding a \DborUtfEightStringValue{} as soon as it encounters a codepoint outside
    an expected range and without checking the UTF-8 encoding of the remaining code points.
}


\subsection{Conformance levels}
%%%%%%%%%%%%%%%%%%%%%%%%%%%%%%%
\label{sec:conformancelevels}

DBOR encoders and decoders do not have to support \emph{all} types of values and \emph{all} their instances
(but as far as they do they must adhere to this specification, naturally).

To allow easy description and comparision of capabilities of encoders and decoders,
certain common capabilities (or constraints) are grouped by \emph{conformance levels}.

\medskip
If a conformance level specifies a value~$v$ as \emph{supported},
a decoder of this conformance level supports $v$ and can decode $v$ into at least one native data type~$T$:
\begin{enumerate}[a)]
    \item successfully, representing $o$ exactly or approximately, or
    \item unsuccessfully, if $T$ cannot represent $o$ at least approximately.
\end{enumerate}

\medskip
If a conformance level specifies a value~$v$ representing an object~$o$ as \emph{supported and representable}
\begin{itemize}
    \item
    an encoder of this conformance level can produce a well-formed~$v$ from at least one native data type
    representing $o$ exactly, provided the available memory is sufficient,

    \item
    a decoder of this conformance level supports~$v$ and can
    decode a (well-formed) $v$ into a at least one native data type representing $o$ exactly,
    provided the available memory is sufficient.
\end{itemize}

\begin{BeginParPenalty}

\paragraph{Supported values in conformance level 0}
%%%%%%%%%%%%%%%%%%%%%%%%%%%%%%%%%%%%%%%%%%%%%%%%%%%

\begin{quote}
    \noindent
    \begin{tabular}{l v{0.4\textwidth}}
        \toprule
        Value & Supported/representable if \\
        \midrule
        \DborNoneValue() &
            supported \\
        \DborIntegerValue($v$) &
            supported and representable:
            $v \in \IntegerInterval{-24}{23}$ \\
        \DborByteStringValue(\ByteSequence{b_1, \ldots, b_m}) &
            supported and representable:
            $m \in \IntegerInterval{0}{23}$ \\
        \DborUtfEightStringValue(\ByteSequence{b_1, \ldots, b_m}) &
            supported and representable:
            $m \in \IntegerInterval{0}{23}$ and $b_i \in \IntegerInterval{0}{\HexNumber{7F}}$
            for all $i \in \IntegerInterval{0}{m}$ (no multi-byte UTF-8 sequences) \\
        \bottomrule
    \end{tabular}
\end{quote}

\paragraph{Supported values in conformance level~$1_M$ with $M \in \{16, 32, 64\}$}
%%%%%%%%%%%%%%%%%%%%%%%%%%%%%%%%%%%%%%%%%%%%%%%%%%%%%%%%%%%%%%%%%%%%%%%%%%%%%%%%%%%

\begin{quote}
    \noindent
    \begin{tabular}{l v{0.4\textwidth}}
        \toprule
        Value & Supported/representable if \\
        \midrule
        \DborNoneValue() &
            supported \\
        \DborIntegerValue($v$) &
            supported;\par
            representable: $v \in \IntegerInterval{-2^{M-1}}{2^M - 1}$ \\
        \DborByteStringValue(\ByteSequence{b_1, \ldots, b_m}) &
            supported;\par
            representable: $m \in \IntegerInterval{0}{2^M - 1}$ \\
        \DborUtfEightStringValue(\ByteSequence{b_1, \ldots, b_m}) &
            supported;\par
            representable:
            $m \in \IntegerInterval{0}{2^M - 1}$ and $b_i \in \IntegerInterval{0}{\HexNumber{7F}}$
            for $i \in \IntegerInterval{0}{m}$ (no multi-byte UTF-8 sequences) \\
        \bottomrule
    \end{tabular}
\end{quote}

\paragraph{Supported values in conformance level 2}
%%%%%%%%%%%%%%%%%%%%%%%%%%%%%%%%%%%%%%%%%%%%%%%%%%%

\begin{quote}
    \noindent
    \begin{tabular}{l v{0.4\textwidth}}
        \toprule
        Value & Supported/representable if \\
        \midrule
        \DborNoneValue() &
            supported \\
        \DborIntegerValue($v$) &
            supported;\par
            representable: $v \in \IntegerInterval{-2^{63}}{2^{64} - 1}$ \\
        \DborByteStringValue(\ByteSequence{b_1, \ldots, b_m}) &
            supported and representable \\
        \DborUtfEightStringValue(\ByteSequence{b_1, \ldots, b_m}) &
            supported and representable \\
        \DborSequenceValue($v_1, \ldots, v_r$) &
            $v_i$ is a supported/representable value of this conformance level for $i \in \IntegerInterval{0}{r}$ \\
        \bottomrule
    \end{tabular}
\end{quote}

\paragraph{Supported values in conformance level 3}
%%%%%%%%%%%%%%%%%%%%%%%%%%%%%%%%%%%%%%%%%%%%%%%%%%%

\begin{quote}
    \noindent
    \begin{tabular}{l v{0.4\textwidth}}
        \toprule
        Value & Supported/representable if \\
        \midrule
        \DborNoneValue() &
            supported \\
        \DborMinusZeroValue() &
            supported and representable \\
        \DborMinusInfinityValue() &
            supported and representable \\
        \DborInfinityValue() &
            supported and representable \\
        \DborIntegerValue($v$) &
            supported;\par
            representable: $v \in \IntegerInterval{-2^{63}}{2^{64} - 1}$ \\
        \DborBinaryRationalValue($p$, $v$) &
            supported and representable;\par
            encoder may always use $p = 52$ (value is not necessarily canonical) \\
        \DborDecimalRationalValue($m$, $e$) &
            supported: $e \in \IntegerInterval{-2^{31}}{2^{31} - 1}$;\par
            representable: $m \in \IntegerInterval{-2^{63}}{2^{63} - 1}$ and $e \in \IntegerInterval{-2^{31}}{2^{31} - 1}$ \\
        \DborByteStringValue(\ByteSequence{b_1, \ldots, b_m}) &
            supported and representable \\
        \DborUtfEightStringValue(\ByteSequence{b_1, \ldots, b_m}) &
            supported and representable \\
        \DborSequenceValue($v_1, \ldots, v_r$) &
            $v_i$ is a supported/representable value of this conformance level for $i \in \IntegerInterval{0}{r}$ \\
        \DborDictionaryValue($k_1 \mapsto v_1, \ldots, k_r \mapsto v_r$) &
            $k_i$, $v_i$ are supported/representable values of this conformance level for $i \in \IntegerInterval{0}{r}$ \\
        \bottomrule
    \end{tabular}
\end{quote}

\paragraph{Supported values in conformance level 5}
%%%%%%%%%%%%%%%%%%%%%%%%%%%%%%%%%%%%%%%%%%%%%%%%%%%

\begin{quote}
    \noindent
    \begin{tabular}{l v{0.4\textwidth}}
        \toprule
        Value & Supported/representable if \\
        \midrule
        \DborNoneValue() &
            supported \\
        \DborMinusZeroValue() &
            supported and representable \\
        \DborMinusInfinityValue() &
            supported and representable \\
        \DborInfinityValue() &
            supported and representable \\
        \DborIntegerValue($v$) &
            supported;\par
            representable: $v \in \IntegerInterval{-2^{63}}{2^{64} - 1}$ \\
        \DborBinaryRationalValue($p$, $v$) &
            supported and representable;\par
            encoder produces canonical values only \\
        \DborDecimalRationalValue($m$, $e$) &
            supported: $e \in \IntegerInterval{-2^{31}}{2^{31} - 1}$;\par
            representable: $m \in \IntegerInterval{-2^{63}}{2^{63} - 1}$ and
                $e \in \IntegerInterval{-2^{31}}{2^{31} - 1}$ \\
        \DborByteStringValue(\ByteSequence{b_1, \ldots, b_m}) &
            supported and representable \\
        \DborUtfEightStringValue(\ByteSequence{b_1, \ldots, b_m}) &
            supported and representable \\
        \DborSequenceValue($v_1, \ldots, v_r$) &
            $v_i$ is a supported/representable value of this conformance level for $i \in \IntegerInterval{0}{r}$ \\
        \DborDictionaryValue($k_1 \mapsto v_1, \ldots, k_r \mapsto v_r$) &
            $k_i$, $v_i$ are supported/representable values of this conformance level
            for $i \in \IntegerInterval{0}{r}$ \\
        \bottomrule
    \end{tabular}
\end{quote}

\end{BeginParPenalty}

\documentclass{dbor-article}

    % mathematical symbols and operators
    \newcommand{\e}{\mathrm{e}}
    \DeclareMathOperator{\BitAnd}{AND}
    \DeclareMathOperator{\BitOr}{OR}
    \newcommand{\SetOfReals}{\mathbbm{R}}
    \newcommand{\SetOfIntegers}{\mathbbm{Z}}
    \newcommand{\HexNumber}[1]{\mathrm{#1}_{16}}

    \newcommand{\DborSyntaxIdent}[1]{\texttt{#1}}
    \newcommand{\DborSyntaxIdentRef}[1]{\hyperlink{sec:def:#1}{\DborSyntaxIdent{#1}}}
        % usage: \DborSyntaxIdentRef{IntegerToken}

    % colour coding of bytes sequences
    \newcommand{\DborFirstByte}[2]{\mathbf{\textcolor{#1}{\HexNumber{#2}}}}
    \newcommand{\DborFirstByteNone}[1]{\DborFirstByte{red}{#1}}
    \newcommand{\DborFirstByteNumber}[1]{\DborFirstByte{black}{#1}}
        % IntegerValue, BinaryRationalValue, DecimalRationalValue, MinusZero, Infinity, MinusInfinity
    \newcommand{\DborFirstByteString}[1]{\DborFirstByte{green}{#1}}
        % ByteStringValue, UnicodeStringValue
    \newcommand{\DborFirstByteSequence}[1]{\DborFirstByte{blue}{#1}}
    \newcommand{\DborFirstByteDictionary}[1]{\DborFirstByte{violet}{#1}}
    \newcommand{\DborNextByte}[1]{\HexNumber{#1}}

    \newenvironment{Note}{\bgroup\small\ignorespaces}{\par\egroup\smallskip}

    \newcommand{\IncludeImageInPlace}[1]{%
        \begin{quotation}%
            \includegraphics[scale=0.8]{#1}%
        \end{quotation}%
    }

    \newcommand{\DborVersion}{\input{repo_wd_version.tex}}
    \title{DBOR specification \DborVersion}
    \author{Daniel Lutz}

\begin{document}

    \maketitle
    \tableofcontents

    \section{Introduction}
    %%%%%%%%%%%%%%%%%%%%%%

    \subsection{Notation}
    %%%%%%%%%%%%%%%%%%%%%

    \noindent
    {%
        \setlength\extrarowheight{0.8ex}%
        \begin{tabular}{@{} p{.25\textwidth} p{.5\textwidth}}
            $\SetOfReals$: &
                the set of all real numbers \\
            $\SetOfIntegers$: &
                the set of all real integers \\
            $[a, b]$: & 
                $\{x \in \SetOfReals{:}\; a \le x \le b\}$ \\
            $[a, b)$: &
                $\{x \in \SetOfReals{:}\; a \le x < b\}$ \\
            $\langle a, b, \ldots\rangle$: & 
                sequence of bytes, starting with byte $a$ \\
            $a \mid b$: & 
                concatenation of bytes sequences $a$, $b$: $a$ followed by $b$ \\
            ill-formed \DborSyntaxIdent{\dots Value}: & 
                byte sequence that looks likes a \DborSyntaxIdent{\dots Value} but actually
                is not one%
                \footnote{
                    Differentiation between well-formed and ill-formed \DborSyntaxIdent{Value}s 
                    can be useful in implementations.
                    This specification describes only well-formed \DborSyntaxIdent{Value}s, however.
                } \\             
        \end{tabular}%
    }


    \section{Values}
    %%%%%%%%%%%%%%%%
    \label{sec:values}

    This section defines the encoding of values as sequences of bytes by reducing them to tokens
    (section \ref{sec:tokens}).


    \subsection{Elementary values}
    %%%%%%%%%%%%%%%%%%%%%%%%%%%%%%

    \subsubsection{\DborSyntaxIdent{NoneValue}}
    %%%%%%%%%%%%%%%%%%%%%%%%%%%%%%%%%%%%%%%%%%%
    \label{sec:def:NoneValue}
    \hypertarget{sec:def:NoneValue}{}

    \paragraph{Representable objects}

    A \DborSyntaxIdent{NoneValue} represents the absence of an actual value (like \texttt{null} or \texttt{None} is some
    programming languages or NaN in IEEE-754:2008).
    It is considered different from any object represented by a \DborSyntaxIdentRef{Value} other than
    \DborSyntaxIdent{NoneValue}.

    \paragraph{Representation}

    An \DborSyntaxIdent{NoneValue} is the \DborSyntaxIdentRef{MinimalToken}($\HexNumber{1F}$).

    \smallskip
    \noindent
    Example:
    \nolinebreak
    \begin{quote}    
        \begin{tabular}{ll}
            \toprule
            Value & Representation \\
            \midrule
            \DborSyntaxIdent{NoneValue} 
                & $\langle \DborFirstByteNone{FF} \rangle$ \\
            \bottomrule
        \end{tabular}
    \end{quote}

    \paragraph{Ambiguity, canonical value}

    There is only one representation of a \DborSyntaxIdent{NoneValue}.
    It is canonical.


    \subsubsection{\DborSyntaxIdent{MinusZeroValue}}
    %%%%%%%%%%%%%%%%%%%%%%%%%%%%%%%%%%%%%%%%%%%%%%%%
    \label{sec:def:MinusZeroValue}
    \hypertarget{sec:def:MinusZeroValue}{}

    \paragraph{Representable objects}

    A \DborSyntaxIdent{MinusZeroValue} represents the "number" $-0$ in the sense of IEEE-754:2008 artihmetics.

    \paragraph{Representation}

    An \DborSyntaxIdent{MinusZeroValue} is the \DborSyntaxIdentRef{MinimalToken}($\HexNumber{1C}$).

    \smallskip
    \noindent
    Example:
    \nolinebreak
    \begin{quote}    
        \begin{tabular}{ll}
            \toprule
            Value & Representation \\
            \midrule
            \DborSyntaxIdent{MinusZeroValue} 
                & $\langle \DborFirstByteNumber{FC} \rangle$ \\
            \bottomrule
        \end{tabular}
    \end{quote}

    \paragraph{Ambiguity, canonical value}

    There is only one representation of a \DborSyntaxIdent{MinusZeroValue}.
    It is canonical.


    \subsubsection{\DborSyntaxIdent{InfinityValue}}
    %%%%%%%%%%%%%%%%%%%%%%%%%%%%%%%%%%%%%%%%%%%%%%%
    \label{sec:def:InfinityValue}
    \hypertarget{sec:def:InfinityValue}{}

    \paragraph{Representable objects}

    A \DborSyntaxIdent{InfinityValue} represents the "number" $\infty$ in the sense of IEEE-754:2008 artihmetics.
    It is considered larger than any object represented by a \DborSyntaxIdentRef{NumberValue} 
    other than \DborSyntaxIdent{InfinityValue}.

    \paragraph{Representation}

    An \DborSyntaxIdent{InfinityValue} is the \DborSyntaxIdentRef{MinimalToken}($\HexNumber{1E}$).

    \smallskip
    \noindent
    Example:
    \nolinebreak
    \begin{quote}    
        \begin{tabular}{ll}
            \toprule
            Value & Representation \\
            \midrule
            \DborSyntaxIdent{InfinityValue} 
                & $\langle \DborFirstByteNumber{FE} \rangle$ \\
            \bottomrule
        \end{tabular}
    \end{quote}

    \paragraph{Ambiguity, canonical value}

    There is only one representation of a \DborSyntaxIdent{InfinityValue}.
    It is canonical.


    \subsubsection{\DborSyntaxIdent{MinusInfinityValue}}
    %%%%%%%%%%%%%%%%%%%%%%%%%%%%%%%%%%%%%%%%%%%%%%%%%%%%
    \label{sec:def:MinusInfinityValue}
    \hypertarget{sec:def:MinusInfinityValue}{}

    \paragraph{Representable objects}

    A \DborSyntaxIdent{MinusInfinityValue} represents the "number" $-\infty$ in the sense of IEEE-754:2008 artihmetics.
    It is considered smaller than any object represented by a \DborSyntaxIdentRef{NumberValue} 
    other than \DborSyntaxIdent{MinusInfinityValue}.

    \paragraph{Representation}

    An \DborSyntaxIdent{MinusInfinityValue} is the \DborSyntaxIdentRef{MinimalToken}($\HexNumber{1D}$).

    \smallskip
    \noindent
    Example:
    \nolinebreak
    \begin{quote}    
        \begin{tabular}{ll}
            \toprule
            Value & Representation \\
            \midrule
            \DborSyntaxIdent{MinusInfinityValue} 
                & $\langle \DborFirstByteNumber{FD} \rangle$ \\
            \bottomrule
        \end{tabular}
    \end{quote}

    \paragraph{Ambiguity, canonical value}

    There is only one representation of a \DborSyntaxIdent{MinusInfinityValue}.
    It is canonical.


    \subsubsection{\DborSyntaxIdent{IntegerValue}($p$)}
    %%%%%%%%%%%%%%%%%%%%%%%%%%%%%%%%%%%%%%%%%%%%%%%%%%%
    \hypertarget{sec:def:IntegerValue}{}

    \paragraph{Representable objects}

    An \DborSyntaxIdent{IntegerValue} can represent exactly any number $v \in \SetOfIntegers \cap [-M, M)$ with
    \begin{equation}
        M := \sum_{k = 0}^8 2^{8k} + 23 \approx 1.003921 \cdot 2^{64}.
    \end{equation}%

    \smallskip
    The minimum representable number is $-18\,519\,084\,246\,547\,628\,312$.
    The maximum representable number is $18\,519\,084\,246\,547\,628\,311$.

    \paragraph{Representation}

    A number $v \in \SetOfIntegers \cap [0, M)$ is represented as \DborSyntaxIdentRef{IntegerToken}(0, $v$).
    A number $v \in \SetOfIntegers \cap [-M, 0)$ is represented as \DborSyntaxIdentRef{IntegerToken}(1, $-v - 1$).

    A sequence of bytes $\langle b_1, \ldots, b_n\rangle$ is a \DborSyntaxIdent{IntegerValue} if and only if
    it is \DborSyntaxIdentRef{IntegerToken}(0, $v$) or a \DborSyntaxIdentRef{IntegerToken}(1, $v$) for an
    appropriate $v$.

    \smallskip
    \noindent
    Examples:
    \nolinebreak
    \begin{quote}    
        \begin{tabular}{ll}
            \toprule
            Value & Representation \\
            \midrule
            %
            \DborSyntaxIdent{IntegerValue}($0$)    
                &  $\langle \DborFirstByteNumber{00} \rangle$ \\
            \DborSyntaxIdent{IntegerValue}($23$)
                &  $\langle \DborFirstByteNumber{17} \rangle$ \\
            \DborSyntaxIdent{IntegerValue}($24$)
                &  $\langle \DborFirstByteNumber{18}, \DborNextByte{00} \rangle$ \\
            \DborSyntaxIdent{IntegerValue}($-100$) 
                &  $\langle \DborFirstByteNumber{38}, \DborNextByte{4B} \rangle$ \\
            \DborSyntaxIdent{IntegerValue}($\HexNumber{FF\,FF\,FF\,FF}$) 
                &  $\langle \DborFirstByteNumber{1B}, \DborNextByte{E7}, \DborNextByte{FE},
                   \DborNextByte{FE}, \DborNextByte{FE} \rangle$ \\
            %
            \bottomrule
        \end{tabular}
    \end{quote}

    \paragraph{Ambiguity, canonical value}

    There is at most one representation of a \DborSyntaxIdent{IntegerValue}($v$) for any $v$.
    It is canonical.


    \subsubsection{\DborSyntaxIdent{BinaryRationalValue}($p$, $v$)}
    %%%%%%%%%%%%%%%%%%%%%%%%%%%%%%%%%%%%%%%%%%%%%%%%%%%%%
    \hypertarget{sec:def:BinaryRationalValue}{}

    \paragraph{Representable objects}

    A \DborSyntaxIdent{BinaryRationalValue} can represent exactly any number
    \begin{equation}
        v = (-1)^s \cdot \sum_{k = -52}^0 m_k s^k \cdot 2^e
    \end{equation}
    with $s, m_k \in \{0, 1\}$, $e \in \SetOfIntegers \cap [2 - 2^{10}, 2^{10}]$ and $v \ne 0$.

    The set of representable numbers comprises all normal and subnormal values of the IEEE-754:2008 type
    \texttt{binary64} except $\pm 0$.

    \smallskip
    The maximum representable number is $(2 - 2^{-52}) \cdot 2^{2^{10}} \approx 3.595\,386 \cdot 10^{308}$.
    The smallest positive representable number is $2^{-52} \cdot 2^{2-2^{10}} = 2^{-1074}
    \approx 4.940\,656 \cdot 10^{-324}$.

    \paragraph{Representation}

    A number $(-1)^s \cdot (1 + M/2^p) \cdot 2^e$ with
    \begin{align*}
        s & \in \{0, 1\} \\
        p & \in \{4, 10, 16, 23, 30, 37, 44, 52\} \\
        M & \in \SetOfIntegers \cap [0, 2^p) \\
        e & \in \SetOfIntegers \cap [\max(0, p - 51) - 2^{r - 1} + 1, 2^{r - 1}) \\
        r & := 8 \lfloor p / 7 \rfloor + 7 - p
    \end{align*}%
    is represented as
    \DborSyntaxIdentRef{BinaryRationalToken}($p$, $0$, $s$, $M$, $e + 2^{r - 1} - 1$).

    A number $(-1)^s \cdot M/2^p \cdot 2^{-1022}$ with $p := 52$, $s \in \{0, 1\}$ and
    $M \in \SetOfIntegers \cap (0, 2^p)$ is represented as
    \DborSyntaxIdentRef{BinaryRationalToken}($p$, $1$, $s$, $M$, $0$).

    \smallskip
    \noindent
    Fore reference, here are the resulting combinations for all possible $p$:
    \nolinebreak
    \begin{quote}
        \newcolumntype{R}{>{$}r<{$}}  % package 'array'
        \begin{tabular}{R R R R >{\hspace{-.8em}$}c<{$\hspace{-.8em}} R}
            \toprule
            k & p & r & & e \\
            \midrule
            0 &  3 &  4 & -3 & \ldots & 4 \\
            1 &  5 & 10 & -15 & \ldots & 16 \\
            2 &  7 & 16 & -63 & \ldots & 64 \\
            3 &  8 & 23 & -127 & \ldots & 128 \\
            4 &  9 & 30 & -255 & \ldots & 256 \\
            5 & 10 & 37 & -511 & \ldots & 512 \\
            6 & 11 & 44 & -1023 & \ldots & 1024 \\
            7 & 11 & 52 & -1022 & \ldots & 1024 \\
            \bottomrule
        \end{tabular}
    \end{quote}

    A sequence of bytes $\langle b_1, \ldots, b_n\rangle$ is a \DborSyntaxIdent{BinaryRationalValue} if and only if
    it is a \DborSyntaxIdentRef{BinaryRationalToken}($p$, $o$, $s$, $M$, $E$) for some
    appropriate $p$, $o$, $s$, $M$ and $E$.

    \smallskip
    \noindent
    Examples:
    \nolinebreak
    \begin{quote}    
        \begin{tabular}{ll}
            \toprule
            Value & Representation \\
            \midrule
            \DborSyntaxIdent{BinaryRationalValue}($3$, $\frac{1}{8}$) 
                &  $\langle \DborFirstByteNumber{D0}, \DborNextByte{00} \rangle$ \\
            \DborSyntaxIdent{BinaryRationalValue}($5$, $-(2^{17} - 2^6)$)%
                \footnote{$-\left(1 + (1 - 2^{-10})\right) \cdot 2^{16}$}
                &  $\langle \DborFirstByteNumber{D1}, \DborNextByte{FF}, \DborNextByte{FF} \rangle$ \\
            \DborSyntaxIdent{BinaryRationalValue}($52$, $2^{-1074}$) 
                &  $\langle \DborFirstByteNumber{D7}, \DborNextByte{01}, \DborNextByte{00},
                                                      \DborNextByte{00}, \DborNextByte{00},$ \\
                &  $                                  \DborNextByte{00}, \DborNextByte{00},
                                                      \DborNextByte{00}, \DborNextByte{00} \rangle$ \\
            \bottomrule
        \end{tabular}
    \end{quote}

    \paragraph{Ambiguity, canonical value}

    There may be more than one representation of \DborSyntaxIdent{Binary\-Rational\-Value}($p$, $v$) for any $v$.
    Of these, the one with the smallest $p$ is the canonical value.%
    \footnote{
        A \DborSyntaxIdent{BinaryRationalValue}($p$, $v$) is therefore canonical if and only if there is
        no \DborSyntaxIdent{BinaryRationalValue}($q$, $v$) with $q < p$.
        See \ref{sec:implementation:BinaryRationalValue:canonical} for an efficient way to check this.
    }
    
    For every \DborSyntaxIdent{Binary\-Rational\-Value}($p$, $v$) with $p < 52$, there is also
    a \DborSyntaxIdent{Binary\-Rational\-Value}($p$, $v$) for any
    $q \in \{4, 10, 16, 23, 30, 37, 44, 52\}$ with $q > p$.

    
    \section{Tokens}
    %%%%%%%%%%%%%%%%    
    \label{sec:tokens}

    \subsection{\DborSyntaxIdent{MinimalToken}($v$)}
    %%%%%%%%%%%%%%%%%%%%%%%%%%%%%%%%%%%%%%%%%%%
    \hypertarget{sec:def:MinimalToken}{}

    A \DborSyntaxIdent{MinimalToken}($v$) for $v \in \SetOfIntegers \cap [0, 2^5)$
    is the following byte:

    \IncludeImageInPlace{MinimalToken}


    \subsection{\DborSyntaxIdent{IntegerToken}($h$, $v$)}
    %%%%%%%%%%%%%%%%%%%%%%%%%%%%%%%%%%%%%%%%%%%%%%
    \hypertarget{sec:def:IntegerToken}{}

    An \DborSyntaxIdent{IntegerToken}($h$, $v$) for $h \in \{0, 1, \ldots, 7\}$ and
    $v \in \SetOfIntegers \cap [0, 24)$ is the following byte:

    \IncludeImageInPlace{IntegerTokenA}
    
    An \DborSyntaxIdent{IntegerToken}($h$, $v$) for $h \in \{0, 1, \ldots, 7\}$ and
    $v \in \SetOfIntegers \cap [24, \sum_{k = 0}^8 2^{8k} + 23)$
    is the following sequence of bytes:

    \IncludeImageInPlace{IntegerTokenB}

    Its $d_i \in \SetOfIntegers \cap [0, 2^8)$ are uniquely%
    \footnote{%
        Since $(v - 23) \bmod 2^{8 j} = \sum_{i = 0}^{j - 1} (d_i + 1) 2^{8 i}$ for every $j$ with $0 < j \le k$,
        the $d_i$ can be calculated with increasing $i$.
    }
    defined by 
    \begin{equation}
        v = 23 + \sum_{i = 0}^k (d_i + 1) 2^{8 i}.
    \end{equation}

    
    \subsection{\DborSyntaxIdent{BinaryRationalToken}($p$, $o$, $s$, $M$, $E$)}
    %%%%%%%%%%%%%%%%%%%%%%%%%%%%%%%%%%%%%%%%%%%%%%%%%%%%%%%%%%%%%%%%%%%%%%%%%%%
    \hypertarget{sec:def:BinaryRationalToken}{}

    A \DborSyntaxIdent{BinaryRationalToken}($p$, $o$, $s$, $M$, $E$) for
    \begin{align*}
        p & \in \{4, 10, 16, 23, 30, 37, 44, 52\} \\
        o & \in \{0, 1\} \\
        s & \in \{0, 1\} \\
        M & \in \SetOfIntegers \cap [0, 2^p) \\
        E & \in \SetOfIntegers \cap [0, 2^r)
            \quad\text{with}\quad r := 8 k + 7 - p
            \quad\text{and}\quad k := \lfloor p / 7 \rfloor
    \end{align*}%
    and
    \begin{align*}
        M = E = 0 \quad & \Rightarrow \quad p < 52 \\
        p < 52 \quad & \Rightarrow \quad o = 0
    \end{align*}%
    is the following sequence of bytes:

    \IncludeImageInPlace{BinaryRationalToken}

    It represents the value
    \begin{equation}
        v = (-1)^s \cdot m \cdot 2^e
    \end{equation}
    with
    \begin{align*}
        m & := 1 - o + M / 2^p \\
        e & := \max(E, o) - 2^{r-1} + 1.
    \end{align*}

    \section{Implementation hints}
    %%%%%%%%%%%%%%%%%%%%%%%%%%%%%%
    \label{sec:implementation}

    \subsection{Check if canonical \DborSyntaxIdent{BinaryRationalValue}}
    %%%%%%%%%%%%%%%%%%%%%%%%%%%%%%%%%%%%%%%%%%%%%%%%%%%%%%%%%%%%%%%%%%%%%
    \label{sec:implementation:BinaryRationalValue:canonical}

    A sequence of bytes $\langle \HexNumber{D0} + k, d_0, \ldots, d_k\rangle$ with
    $k \in \{0, \ldots, 7\}$ and $i := \lfloor \frac{k + 1}{4} \rfloor$ 
    is a \DborSyntaxIdent{Binary\-Rational\-Value} that is canonical if and only if
    \begin{itemize}
        \item
        $k = 0$ or then

        \item
        $2^{2 - i} d_0 \bmod 2^8 \ne 0$ or then

        \item
        If $k < 7$:
        $(d_{ke} < 2 - i) \vee (d_{ke} > 4 + i)$
        with $d_{ke} := \lfloor d_k / 2^4 \rfloor \bmod 8$.

        If $k = 7$:
        $(d_k \BitAnd \HexNumber{7F} = 0) \wedge (d_{k - 1} \BitAnd \HexNumber{F0} = 0)
        \wedge (d_0 \BitOr \cdots \BitOr d_{k - 1} \ne 0)$
    \end{itemize}

    \subsection{Check if \DborSyntaxIdent{IntegerValue} represents multiple of 10}
    %%%%%%%%%%%%%%%%%%%%%%%%%%%%%%%%%%%%%%%%%%%%%%%%%%%%%%%%%%%%%%%%%%%%%%%%%%%%%%
    \label{sec:implementation:IntegerValue:10}

    An integer $v$ is an integer multiple of $10$ if and only if it is an integer multiple
    of $2$ and of $5$.
    Since $2^{8i} \bmod 2 = 0$ for $i \in \{0, 1, \ldots\}$ 
    and $2^{8i} \bmod 5 = 1$ for $i \in \{1, 2, \ldots\}$, 
    a division of $v$ can be replaced by divisions of fewer bytes.

    Example:
    Let $\langle b, d_0, \ldots, d_k\rangle$ be \DborSyntaxIdent{IntegerValue}($v$) for $v \ge 24$.
    Then
    \begin{align*}
        v \bmod 2
            & = \big(23 + \sum_{i = 0}^k (d_i + 1) 2^{8 i}\big) \bmod 2
            = d_0 \bmod 2 \\
        v \bmod 5
            & = \big(23 + \sum_{i = 0}^k (d_i + 1) 2^{8 i}\big) \bmod 5
            = \big(\sum_{i = 0}^k (d_i \bmod 5) + k + 4\big) \bmod 5
    \end{align*}

\end{document}

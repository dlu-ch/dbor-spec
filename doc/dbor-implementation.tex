% SPDX-License-Identifier: CC-BY-4.0
% DBOR specification - Dense Binary Object Representation
% Copyright (C) 2020 Daniel Lutz <dlu-ch@users.noreply.github.com>

\section{Implementation hints}
%%%%%%%%%%%%%%%%%%%%%%%%%%%%%%
\label{sec:implementation}

\subsection{Representation overview by first byte}
%%%%%%%%%%%%%%%%%%%%%%%%%%%%%%%%%%%%%%%%%%%%%%%%%%
\label{sec:implementation:representation_by_first_byte}

Values:\nolinebreak
\begin{quote}
    \noindent
    \newcolumntype{L}{>{$}l<{$}}% package 'array'
    \setlength\extrarowheight{0.7ex}
    \begin{tabular}{L v{0.5\textwidth} L}
        \toprule
        \text{First byte} & Value type & \text{Constraints} \\
        \midrule
        \BinNumber{000x\,xxxx}
            & \DborIntegerValue*($v$)
            & v \ge 0 \\
        \BinNumber{001x\,xxxx}
            & \DborIntegerValue*($-v - 1$)
            & v < 0 \\
        \BinNumber{010x\,xxxx}
            & \DborByteStringValue(\dots) \\
        \BinNumber{011x\,xxxx}
            & \DborUtfEightStringValue(\dots) \\
        \BinNumber{100x\,xxxx}
            & \DborSequenceValue*(\dots) \\
        \BinNumber{101x\,xxxx}
            & \DborDictionaryValue*(\dots) \\
        \BinNumber{1100\,0xxx}
            & \DborAllocatedValue*(\dots) \\
        \BinNumber{1100\,1xxx}
            & \DborBinaryRationalValue*(\dots) \\
        \BinNumber{1101\,xxxx}
            & \DborDecimalRationalValue*($m$, $e$)
            & |e| > 8 \\
        \BinNumber{1110\,xxxx}
            & \DborDecimalRationalValue*($m$, $e$)
            & |e| \le 8 \\
        \BinNumber{1111\,0xxx}
            & reserved \\
        \BinNumber{1111\,10xx}
            & reserved \\
        \BinNumber{1111\,11xx}
            & \DborNoneValue*, \DborMinusZeroValue*, \DborInfinityValue*, \DborMinusInfinityValue* \\
        \bottomrule
    \end{tabular}
\end{quote}

\medskip
Tokens:\nolinebreak
\begin{quote}
    \noindent
    \newcolumntype{L}{>{$}l<{$}}% package 'array'
    \setlength\extrarowheight{0.7ex}
    \begin{tabular}{L v{0.5\textwidth} L}
        \toprule
        \text{First byte} & Token representation & \text{Constraints} \\
        \midrule
        \BinNumber{000x\,xxxx}
            & \DborIntegerToken*(0, $v$)
            & v \ge 0 \\
        \BinNumber{001x\,xxxx}
            & \DborIntegerToken*(1, $-v - 1$)
            & v < 0 \\
        \BinNumber{010x\,xxxx}
            & \DborIntegerToken*($2$, $m$) {\Concat} \ByteSequence{b_1, \ldots, b_m} \\
        \BinNumber{011x\,xxxx}
            & \DborIntegerToken*($3$, $m$) {\Concat} \ByteSequence{b_1, \ldots, b_m} \\
        \BinNumber{100x\,xxxx}
            & \DborIntegerToken*($4$, $m$) {\Concat} $v_1 \Concat \cdots \Concat v_r$ \\
        \BinNumber{101x\,xxxx}
            & \DborIntegerToken*($5$, $m$) {\Concat} $k_1 \Concat v_1 \Concat \cdots \Concat k_r \Concat v_r$ \\
        \BinNumber{1100\,0xxx}
            & \DborNaturalToken*($6$, $0$, $m$) {\Concat} $v \Concat \ByteSequence{f_{\|v\| + 1}, \ldots, f_m}$ \\
        \BinNumber{1100\,1xxx}
            & \DborBinaryRationalToken*(\dots) \\
        \BinNumber{1101\,xxxx}
            & \DborPowerOfTenToken*($e$) {\Concat} \DborIntegerValue*(\dots)
            & |e| > 8 \\
        \BinNumber{1110\,xxxx}
            & \DborPowerOfTenToken*($e$) {\Concat} \DborIntegerValue*(\dots)
            & |e| \le 8 \\
        \BinNumber{1111\,0xxx}
            & \DborMinimalToken*($7$, $v$)
            & \HexNumber{10} \le v < \HexNumber{1B} \\
        \BinNumber{1111\,10xx}
            & \DborMinimalToken*($7$, $v$)
            & \HexNumber{1B} \le v < \HexNumber{1C} \\
        \BinNumber{1111\,11xx}
            & \DborMinimalToken*($7$, $v$)
            & v \ge \HexNumber{1C} \\
        \bottomrule
    \end{tabular}
\end{quote}



\subsection{Check if canonical \DborBinaryRationalValue}
%%%%%%%%%%%%%%%%%%%%%%%%%%%%%%%%%%%%%%%%%%%%%%%%%%%%%%%%
\label{sec:implementation:BinaryRationalValue:canonical}

A byte sequence \ByteSequence{\HexNumber{D0} + k, d_0, \ldots, d_k} with
$k \in \IntegerInterval{0}{7}$ and $i := \lfloor \frac{k + 1}{4} \rfloor$
is a \DborBinaryRationalValue{} that is canonical if and only if
\begin{itemize}
    \item
    $k = 0$ or then

    \item
    $2^{2 - i} d_0 \bmod 2^8 \ne 0$ or then

    \item
    If $k < 7$:
    $(d_{ke} < 2 - i) \vee (d_{ke} > 4 + i)$
    with $d_{ke} := \lfloor d_k / 2^4 \rfloor \bmod 8$.

    If $k = 7$:
    $(d_k \BitAnd \HexNumber{7F} = 0) \wedge (d_{k - 1} \BitAnd \HexNumber{F0} = 0)
    \wedge (d_0 \BitOr \cdots \BitOr d_{k - 1} \ne 0)$
\end{itemize}


\subsection{Check if \DborIntegerValue{} represents multiple of 10}
%%%%%%%%%%%%%%%%%%%%%%%%%%%%%%%%%%%%%%%%%%%%%%%%%%%%%%%%%%%%%%%%%%%
\label{sec:implementation:IntegerValue:mod10}

An integer $v$ is an integer multiple of $10$ if and only if it is an integer multiple
of $2$ and of $5$.
Since $2^{8i} \bmod 2 = 0$ for $i \in \{0, 1, \ldots\}$
and $2^{8i} \bmod 5 = 1$ for $i \in \{1, 2, \ldots\}$,
a division of $v$ can be replaced by divisions of fewer bytes.

Example:
Let $\langle b, d_0, \ldots, d_k\rangle$ be \DborIntegerValue($v$) for $v \ge 24$.
Then
\begin{align*}
    v \bmod 2
        & = \big(23 + \sum_{i = 0}^k (d_i + 1) 2^{8 i}\big) \bmod 2
        = d_0 \bmod 2 \\
    v \bmod 5
        & = \big(23 + \sum_{i = 0}^k (d_i + 1) 2^{8 i}\big) \bmod 5
        = \big(\sum_{i = 0}^k (d_i \bmod 5) + k + 4\big) \bmod 5
\end{align*}

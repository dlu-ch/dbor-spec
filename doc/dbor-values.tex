\section{Values}
%%%%%%%%%%%%%%%%
\label{sec:values}

This section defines the encoding of values as sequences of bytes by reducing them to tokens
(section \ref{sec:tokens}).


\subsection{Elementary values}
%%%%%%%%%%%%%%%%%%%%%%%%%%%%%%

\subsubsection{\DborSyntaxIdent{NoneValue}}
%%%%%%%%%%%%%%%%%%%%%%%%%%%%%%%%%%%%%%%%%%%
\label{sec:def:NoneValue}
\hypertarget{sec:def:NoneValue}{}

\paragraph{Representable objects}

A \DborSyntaxIdent{NoneValue} represents the absence of an actual value (like \texttt{null} or \texttt{None} is some
programming languages or NaN in IEEE-754:2008).
It is considered different from any object represented by a \DborSyntaxIdentRef{Value} other than
\DborSyntaxIdent{NoneValue}.

\paragraph{Representation}

An \DborSyntaxIdent{NoneValue} is the \DborSyntaxIdentRef{MinimalToken}($\HexNumber{1F}$).

\smallskip
\noindent
Example:
\nolinebreak
\begin{quote}
    \begin{tabular}{ll}
        \toprule
        Value & Representation \\
        \midrule
        \DborSyntaxIdent{NoneValue}
            & $\langle \DborFirstByteNone{FF} \rangle$ \\
        \bottomrule
    \end{tabular}
\end{quote}

\paragraph{Ambiguity, canonical value}

There is only one representation of a \DborSyntaxIdent{NoneValue}.
It is canonical.


\subsubsection{\DborSyntaxIdent{MinusZeroValue}}
%%%%%%%%%%%%%%%%%%%%%%%%%%%%%%%%%%%%%%%%%%%%%%%%
\label{sec:def:MinusZeroValue}
\hypertarget{sec:def:MinusZeroValue}{}

\paragraph{Representable objects}

A \DborSyntaxIdent{MinusZeroValue} represents the "number" $-0$ in the sense of IEEE-754:2008 artihmetics.

\paragraph{Representation}

An \DborSyntaxIdent{MinusZeroValue} is the \DborSyntaxIdentRef{MinimalToken}($\HexNumber{1C}$).

\smallskip
\noindent
Example:
\nolinebreak
\begin{quote}
    \begin{tabular}{ll}
        \toprule
        Value & Representation \\
        \midrule
        \DborSyntaxIdent{MinusZeroValue}
            & $\langle \DborFirstByteNumber{FC} \rangle$ \\
        \bottomrule
    \end{tabular}
\end{quote}

\paragraph{Ambiguity, canonical value}

There is only one representation of a \DborSyntaxIdent{MinusZeroValue}.
It is canonical.


\subsubsection{\DborSyntaxIdent{InfinityValue}}
%%%%%%%%%%%%%%%%%%%%%%%%%%%%%%%%%%%%%%%%%%%%%%%
\label{sec:def:InfinityValue}
\hypertarget{sec:def:InfinityValue}{}

\paragraph{Representable objects}

A \DborSyntaxIdent{InfinityValue} represents the "number" $\infty$ in the sense of IEEE-754:2008 artihmetics.
It is considered larger than any object represented by a \DborSyntaxIdentRef{NumberValue}
other than \DborSyntaxIdent{InfinityValue}.

\paragraph{Representation}

An \DborSyntaxIdent{InfinityValue} is the \DborSyntaxIdentRef{MinimalToken}($\HexNumber{1E}$).

\smallskip
\noindent
Example:
\nolinebreak
\begin{quote}
    \begin{tabular}{ll}
        \toprule
        Value & Representation \\
        \midrule
        \DborSyntaxIdent{InfinityValue}
            & $\langle \DborFirstByteNumber{FE} \rangle$ \\
        \bottomrule
    \end{tabular}
\end{quote}

\paragraph{Ambiguity, canonical value}

There is only one representation of a \DborSyntaxIdent{InfinityValue}.
It is canonical.


\subsubsection{\DborSyntaxIdent{MinusInfinityValue}}
%%%%%%%%%%%%%%%%%%%%%%%%%%%%%%%%%%%%%%%%%%%%%%%%%%%%
\label{sec:def:MinusInfinityValue}
\hypertarget{sec:def:MinusInfinityValue}{}

\paragraph{Representable objects}

A \DborSyntaxIdent{MinusInfinityValue} represents the "number" $-\infty$ in the sense of IEEE-754:2008 artihmetics.
It is considered smaller than any object represented by a \DborSyntaxIdentRef{NumberValue}
other than \DborSyntaxIdent{MinusInfinityValue}.

\paragraph{Representation}

An \DborSyntaxIdent{MinusInfinityValue} is the \DborSyntaxIdentRef{MinimalToken}($\HexNumber{1D}$).

\smallskip
\noindent
Example:
\nolinebreak
\begin{quote}
    \begin{tabular}{ll}
        \toprule
        Value & Representation \\
        \midrule
        \DborSyntaxIdent{MinusInfinityValue}
            & $\langle \DborFirstByteNumber{FD} \rangle$ \\
        \bottomrule
    \end{tabular}
\end{quote}

\paragraph{Ambiguity, canonical value}

There is only one representation of a \DborSyntaxIdent{MinusInfinityValue}.
It is canonical.


\subsubsection{\DborSyntaxIdent{IntegerValue}($p$)}
%%%%%%%%%%%%%%%%%%%%%%%%%%%%%%%%%%%%%%%%%%%%%%%%%%%
\hypertarget{sec:def:IntegerValue}{}

\paragraph{Representable objects}

An \DborSyntaxIdent{IntegerValue} can represent exactly any number $v \in \SetOfIntegers \cap [-M, M)$ with
\begin{equation}
    M := \sum_{k = 0}^8 2^{8k} + 23 \approx 1.003921 \cdot 2^{64}.
\end{equation}%

\smallskip
The minimum representable number is $-18\,519\,084\,246\,547\,628\,312$.
The maximum representable number is $18\,519\,084\,246\,547\,628\,311$.

\paragraph{Representation}

A number $v \in \SetOfIntegers \cap [0, M)$ is represented as \DborSyntaxIdentRef{IntegerToken}(0, $v$).
A number $v \in \SetOfIntegers \cap [-M, 0)$ is represented as \DborSyntaxIdentRef{IntegerToken}(1, $-v - 1$).

A sequence of bytes $\langle b_1, \ldots, b_n\rangle$ is a \DborSyntaxIdent{IntegerValue} if and only if
it is \DborSyntaxIdentRef{IntegerToken}(0, $v$) or a \DborSyntaxIdentRef{IntegerToken}(1, $v$) for an
appropriate $v$.

\smallskip
\noindent
Examples:
\nolinebreak
\begin{quote}
    \begin{tabular}{ll}
        \toprule
        Value & Representation \\
        \midrule
        %
        \DborSyntaxIdent{IntegerValue}($0$)
            &  $\langle \DborFirstByteNumber{00} \rangle$ \\
        \DborSyntaxIdent{IntegerValue}($23$)
            &  $\langle \DborFirstByteNumber{17} \rangle$ \\
        \DborSyntaxIdent{IntegerValue}($24$)
            &  $\langle \DborFirstByteNumber{18}, \DborNextByte{00} \rangle$ \\
        \DborSyntaxIdent{IntegerValue}($-100$)
            &  $\langle \DborFirstByteNumber{38}, \DborNextByte{4B} \rangle$ \\
        \DborSyntaxIdent{IntegerValue}($\HexNumber{FF\,FF\,FF\,FF}$)
            &  $\langle \DborFirstByteNumber{1B}, \DborNextByte{E7}, \DborNextByte{FE},
               \DborNextByte{FE}, \DborNextByte{FE} \rangle$ \\
        %
        \bottomrule
    \end{tabular}
\end{quote}

\paragraph{Ambiguity, canonical value}

There is at most one representation of a \DborSyntaxIdent{IntegerValue}($v$) for any $v$.
It is canonical.


\subsubsection{\DborSyntaxIdent{BinaryRationalValue}($p$, $v$)}
%%%%%%%%%%%%%%%%%%%%%%%%%%%%%%%%%%%%%%%%%%%%%%%%%%%%%
\hypertarget{sec:def:BinaryRationalValue}{}

\paragraph{Representable objects}

A \DborSyntaxIdent{BinaryRationalValue} can represent exactly any number
\begin{equation}
    v = (-1)^s \cdot \sum_{k = -52}^0 m_k s^k \cdot 2^e
\end{equation}
with $s, m_k \in \{0, 1\}$, $e \in \SetOfIntegers \cap [2 - 2^{10}, 2^{10}]$ and $v \ne 0$.

The set of representable numbers comprises all normal and subnormal values of the IEEE-754:2008 type
\texttt{binary64} except $\pm 0$.

\smallskip
The maximum representable number is $(2 - 2^{-52}) \cdot 2^{2^{10}} \approx 3.595\,386 \cdot 10^{308}$.
The smallest positive representable number is $2^{-52} \cdot 2^{2-2^{10}} = 2^{-1074}
\approx 4.940\,656 \cdot 10^{-324}$.

\paragraph{Representation}

A number $(-1)^s \cdot (1 + M/2^p) \cdot 2^e$ with
\begin{align*}
    s & \in \{0, 1\} \\
    p & \in \{4, 10, 16, 23, 30, 37, 44, 52\} \\
    M & \in \SetOfIntegers \cap [0, 2^p) \\
    e & \in \SetOfIntegers \cap [\max(0, p - 51) - 2^{r - 1} + 1, 2^{r - 1}) \\
    r & := 8 \lfloor p / 7 \rfloor + 7 - p
\end{align*}%
is represented as
\DborSyntaxIdentRef{BinaryRationalToken}($p$, $0$, $s$, $M$, $e + 2^{r - 1} - 1$).

A number $(-1)^s \cdot M/2^p \cdot 2^{-1022}$ with $p := 52$, $s \in \{0, 1\}$ and
$M \in \SetOfIntegers \cap (0, 2^p)$ is represented as
\DborSyntaxIdentRef{BinaryRationalToken}($p$, $1$, $s$, $M$, $0$).

\smallskip
\noindent
Fore reference, here are the resulting combinations for all possible $p$:
\nolinebreak
\begin{quote}
    \newcolumntype{R}{>{$}r<{$}}  % package 'array'
    \begin{tabular}{R R R R >{\hspace{-.8em}$}c<{$\hspace{-.8em}} R}
        \toprule
        k & p & r & & e \\
        \midrule
        0 &  3 &  4 & -3 & \ldots & 4 \\
        1 &  5 & 10 & -15 & \ldots & 16 \\
        2 &  7 & 16 & -63 & \ldots & 64 \\
        3 &  8 & 23 & -127 & \ldots & 128 \\
        4 &  9 & 30 & -255 & \ldots & 256 \\
        5 & 10 & 37 & -511 & \ldots & 512 \\
        6 & 11 & 44 & -1023 & \ldots & 1024 \\
        7 & 11 & 52 & -1022 & \ldots & 1024 \\
        \bottomrule
    \end{tabular}
\end{quote}

A sequence of bytes $\langle b_1, \ldots, b_n\rangle$ is a \DborSyntaxIdent{BinaryRationalValue} if and only if
it is a \DborSyntaxIdentRef{BinaryRationalToken}($p$, $o$, $s$, $M$, $E$) for some
appropriate $p$, $o$, $s$, $M$ and $E$.

\smallskip
\noindent
Examples:
\nolinebreak
\begin{quote}
    \begin{tabular}{ll}
        \toprule
        Value & Representation \\
        \midrule
        \DborSyntaxIdent{BinaryRationalValue}($3$, $\frac{1}{8}$)
            &  $\langle \DborFirstByteNumber{D0}, \DborNextByte{00} \rangle$ \\
        \DborSyntaxIdent{BinaryRationalValue}($5$, $-(2^{17} - 2^6)$)%
            \footnote{$-\left(1 + (1 - 2^{-10})\right) \cdot 2^{16}$}
            &  $\langle \DborFirstByteNumber{D1}, \DborNextByte{FF}, \DborNextByte{FF} \rangle$ \\
        \DborSyntaxIdent{BinaryRationalValue}($52$, $2^{-1074}$)
            &  $\langle \DborFirstByteNumber{D7}, \DborNextByte{01}, \DborNextByte{00},
                                                  \DborNextByte{00}, \DborNextByte{00},$ \\
            &  $                                  \DborNextByte{00}, \DborNextByte{00},
                                                  \DborNextByte{00}, \DborNextByte{00} \rangle$ \\
        \bottomrule
    \end{tabular}
\end{quote}

\paragraph{Ambiguity, canonical value}

There may be more than one representation of \DborSyntaxIdent{Binary\-Rational\-Value}($p$, $v$) for any $v$.
Of these, the one with the smallest $p$ is the canonical value.%
\footnote{
    A \DborSyntaxIdent{BinaryRationalValue}($p$, $v$) is therefore canonical if and only if there is
    no \DborSyntaxIdent{BinaryRationalValue}($q$, $v$) with $q < p$.
    See \ref{sec:implementation:BinaryRationalValue:canonical} for an efficient way to check this.
}

For every \DborSyntaxIdent{Binary\-Rational\-Value}($p$, $v$) with $p < 52$, there is also
a \DborSyntaxIdent{Binary\-Rational\-Value}($p$, $v$) for any
$q \in \{4, 10, 16, 23, 30, 37, 44, 52\}$ with $q > p$.

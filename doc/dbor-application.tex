% SPDX-License-Identifier: CC-BY-4.0
% DBOR specification - Dense Binary Object Representation
% Copyright (C) 2020 Daniel Lutz <dlu-ch@users.noreply.github.com>

\section{Application recommendations}
%%%%%%%%%%%%%%%%%%%%%%%%%%%%%%%%%%%%%
\label{sec:applicationrecommendations}

\subsection{\DborUtfEightStringValue}
%%%%%%%%%%%%%%%%%%%%%%%%%%%%%%%%%%%%%

The recommended form for a Unicode string is
Normalization Form C (\href{https://www.unicode.org/versions/Unicode13.0.0/ch03.pdf\#G31703}{NFC}).%
\footnote{
    Python 3: \texttt{unicodedata.normalize('NFC', \dots)}
}
A string should only contain code points from one of the Private Use Areas if their meaning is clearly specified.


\subsection{\DborDictionaryValue}
%%%%%%%%%%%%%%%%%%%%%%%%%%%%%%%%%

The keys of a \DborDictionaryValue{} are formed by the bytes sequences, \emph{not} the represented objects.
It is therefore recommended to use only canonical values as keys.
\DborIntegerValue{}, \DborByteStringValue{}, and \DborUtfEightStringValue{} with code points $< \HexNumber{300}$
are especially well suited;
they are all canonical and can easily be ordered according to section~\ref{sec:stricttotalorder} \emph{before} encoding.


\subsection{Derived types}
%%%%%%%%%%%%%%%%%%%%%%%%%%

DBOR is not extensible (on purpose) and keeps away from attaching context-dependent semantics to the represented objects;
from DBOR's perspective, a string is just a string, whether it denotes a date, an enumeration literal,
or the programmer's name.
How DBOR values are used to represent objects that
are more restricted than the underlying DBOR value or
are constructed from more than one DBOR value
is entirely up to the application.

The following sections give some recommendations to start with.


%%%%%%%%%%%%%%%%%%%%%%%%
\subsubsection{Booleans}

The objects \texttt{False} and \texttt{True} can be seen as instances of 1-bit integers and should be encoded as
\DborIntegerValue(0) and \DborIntegerValue(1), respectively.

Consider using enumerations or sets instead of booleans for the representation of options.


%%%%%%%%%%%%%%%%%%%%%%%%%%%%
\subsubsection{Enumerations}

The specific enumerals from a fixed set of enumerals should be encoded as
\begin{enumerate}[a)]
    \item
    distinct, consecutive non-negative \DborIntegerValue{} instances, or

    \item
    distinct, non-empty \DborUtfEightStringValue{} instances with all code points $< \HexNumber{300}$,
    preferably in the form of words from upper-case ASCII letters and digits only, separated by "\_".
\end{enumerate}


%%%%%%%%%%%%%%%%%%%%
\subsubsection{Sets}

Sets should be \DborSequenceValue{} instances of elementary values,
duplicate-free and ordered ascendingly by ${\prec}$ as defined in section~\ref{sec:stricttotalorder}
(like the keys of \DborDictionaryValue).


%%%%%%%%%%%%%%%%%%%%%%%%%%%%%%%
\subsubsection{Times and dates}

Absolute instants in time should be expressed as (proleptic) Gregorian dates and times of day in a
24~hour timekeeping system with a stated relation to the Coordinated Universal Time (UTC).
Such an instant should be represented as a \DborUtfEightStringValue{} in the ISO~8601:2019%
\footnote{%
    Approximated by but not identical to \href{https://tools.ietf.org/html/rfc3339}{RFC~2229}.
} basic format.
Example: "12910801T190000.3Z" for August 1st of the year 1291, 19:00:00.3 UTC
(July 25th in the Julian calendar of the time).
